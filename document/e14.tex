\documentclass[a4paper,12pt]{article}
\usepackage{cite}
\usepackage[dvips]{color}
\usepackage[dvips]{graphicx}
\usepackage{fancyhdr}
\usepackage{here}
\usepackage{threeparttable}
\usepackage{multirow}
\usepackage{amssymb}
\usepackage{amsmath}
\usepackage{mathrsfs}
\usepackage[FIGTOPCAP,nooneline]{subfigure}
\usepackage{boxedminipage}
\usepackage{ascmac}
\usepackage{multicol}


%\setlength{\topmargin}{-5pt}
%\setlength{\oddsidemargin}{0cm}
%\setlength{\evensidemargin}{0cm}
%\setlength{\textwidth}{16cm}
%\setlength{\textheight}{24cm}
\setlength{\belowcaptionskip}{5pt}
\setcounter{tocdepth}{6}

\pagestyle{fancy}
\fancyhf{}
%\fancyhead[er]{\fancyplain{}{\sl\bf\leftmark}}
%\fancyhead[ol]{\fancyplain{}{\sl\bf\rightmark}}
\fancyhead[er]{{\sl\bf\leftmark}}
\fancyhead[ol]{{\sl\bf\rightmark}}
\fancyhead[el,or]{\rm\bf\thepage}
\fancypagestyle{plain} {
\fancyhead[er,ol]{}
\fancyhead[el,or]{\rm\bf\thepage}}
\renewcommand{\headrulewidth}{0.5pt}


\newenvironment{widelineskip}{\addtolength{\baselineskip}{24pt}}{\relax}
\begin{document}
\title{E14 Manual}
\author{Hajime Nanjo}
\maketitle

 \section{Introduction}
  This library contains the software framework for the E14 experiment.
 \section{Supported platform}
  Linux on PC / Mac\\
  For Mac, DYLD\_LIBRARY\_PATH should be set as same as LD\_LIBRARY\_PATH.
  Re-install readline library. Other procedure is the same.
 \section{Requirements}
  \begin{itemize}
   \item subversion\\
         Many recent Linux distribution have subversion already.
   \item ROOT library ($\ge 5.14$)
	 \begin{itemize}
	  \item  CERN lib (optional)
	 \end{itemize}
   \item Geant4 ($\ge 8.3$)
	 \begin{itemize}
	  \item CLHEP corresponding to the Geant4
	 \end{itemize}
   \item readline, history, and, curses libraray\\
          (standard for linux. for mac, re-install readline library.) 
   \item GNU make (standard for linux) 
  \end{itemize}

 \section{Installation on PC Linux}
  \subsection{At KEK RSC}
  \begin{enumerate}
   \item Login RSC and setup for subversion\\
	 For bash user, add lines below to .bashrc.
	 \begin{verbatim}
	SVN=/e14/local/subversion/1.1.4
	# if KEK CC, SVN=/nfs/g/ps/klea/e14/local/subversion/1.1.4
	export LD_LIBRARY_PATH=${SVN}/lib:${LD_LIBRARY_PATH}
	export PATH=${SVN}/bin:${PATH}
	export MANPATH=${SVN}/man:${MANPATH}	
	 \end{verbatim}
   \item Setup ROOT/Geant4/CLHEP
	 \begin{verbatim}
        export ROOTSYS=xxx
        export PATH=${ROOTSYS}/bin:$PATH
        export LD_LIBRARY_PATH=${ROOTSYS}/lib:${LD_LIBRARY_PATH}
        export G4INSTALL=yyy
	source $G4INSTALL/env.sh
	export PATH=${CLHEP_BASE_DIR}/bin:$PATH
	 \end{verbatim}
   \item cd to some directory
	 \begin{verbatim}
	cd ~/a/b/c	
	 \end{verbatim}
	 \~/a/b/c is a directroy you like.
   \item Execute e14export.
	 \begin{verbatim}
	/e14/scripts/e14export e14
	 \end{verbatim}
	 Here you can choose one sub-package instead.
	 \begin{verbatim}
	/e14/scripts/e14export novice01
	 \end{verbatim}
	 All required packages are automatically exported.
   \item Set environments\\
	 For bash user
	 \begin{verbatim}
	cd pro
	source setup.sh	
	 \end{verbatim}
	 This setup.sh is automatically generated and can be used from
	 .bashrc.
   \item Make
	 \begin{verbatim}
	cd e14
	make install-headers
	 \end{verbatim}
	 Please ignore some messages at this stage.
	 \begin{verbatim}
	make
	 \end{verbatim}
  \end{enumerate}
  \subsection{KEK CC}
  \begin{enumerate}
   \item Setup ROOT/Geant4/CLHEP
	 At KEK CC, each version of ROOT is installed under
	 /software/ROOT/,
	 each version of Geant4 is installed under /software/Geant4/,
	 each version of CLHEP is installed under /software/CLHEP/.
	 \begin{verbatim}
        export ROOTSYS=xxx
        export PATH=${ROOTSYS}/bin:$PATH
        export LD_LIBRARY_PATH=${ROOTSYS}/lib:${LD_LIBRARY_PATH}
        export G4INSTALL=yyy
	export G4SYSTEM=zzz
	export G4LIB=xyz 
	# ${G4INSTALL}/lib as usual, but ${G4INSTALL}/slib at KEK CC
	export G4LIB_BUILD_SHARED=1
	export LD_LIBRARY_PATH=${G4LIB}/${G4SYSTEM}:${LD_LIBRARY_PATH}
	export G4LEVELGAMMADATA=$G4INSTALL/../data/....
	# also for other G4 data
	export CLHEP_BASE_DIR=zxy
	export PATH=${CLHEP_BASE_DIR}/bin:$PATH
	export LD_LIBRARY_PATH=${CLHEP_BASE_DIR}/lib:${LD_LIBRARY_PATH}
	 \end{verbatim}
   \item cd to some directory
	 \begin{verbatim}
	cd ~/a/b/c	
	 \end{verbatim}
	 \~/a/b/c is a directroy you like.
   \item Copy the source of E14 library
	 Get the source of E14 library with e14export in KEK RSC or other
	 hosts outside KEK.
	 Copy the source to KEK CC with scp.
	 Edit setup.sh. 
	 The same procedure as ``Set environments'' and the following as in
	 KEK RSC. 
  \end{enumerate}
  \subsection{Outside KEK}
  \begin{enumerate}
   \item Please check svn can be used at your local host.\\
	 \begin{verbatim}
	svn help
	\end{verbatim}	
   \item (optional) Once login RSC and setup for subversion\\
	 This is needed only for developers.\\
	 For bash user, 
           add lines below to .bashrc.
	 \begin{verbatim}
	SVN=/e14/local/subversion/1.1.4
	export LD_LIBRARY_PATH=${SVN}/lib:${LD_LIBRARY_PATH}
	export PATH=${SVN}/bin:${PATH}
	export MANPATH=${SVN}/man:${MANPATH}	
	 \end{verbatim}
   \item Once login RSC and copy scripts to your local host.
	 \begin{verbatim}
	scp -r /e14/scripts xxx.yyy.zzz:~/a/b/c/
	exit #logout
	 \end{verbatim}
   \item Set variables at your local host\\
	 For bash user,
	 \begin{verbatim}
	export E14_GATE_USER=[your account on kl-gate]
	export E14_SERVER_USER=[your account on RSC]
	export E14_SVNHTTP_TUNNEL_PORT=61088
	export E14_SVNSSH_TUNNEL_PORT=61022
	export SVN_SSH="ssh -l ${E14_SERVER_USER} -oHostName=localhost -oHostKeyAlias=chiaki.kek.jp -oPort=${E14_SVNSSH_TUNNEL_PORT}"
	 \end{verbatim}
	 The SVN\_SSH is "ssh -l \$\{E14\_SERVER\_USER\}
	 -oHostName=localhost -oHostKeyAlias=chiaki.kek.jp
	 -oPort=\$\{E14\_SVNSSH\_TUNNEL\_PORT\}"\\
   \item Execute e14connection at your host.
	 \begin{verbatim}
	cd ~/a/b/c/
	./scripts/e14connection start
	 \end{verbatim}
	 Type password for ssh tunnel.
   \item Execute e14export
	 \begin{verbatim}
	./scripts/e14export e14
	 \end{verbatim}
	 Here you can choose one sub-package instead.
	 \begin{verbatim}
	./scripts/e14export novice01
	 \end{verbatim}
	 All required packages are automatically exported.
   \item Set environments\\
	 \begin{verbatim}
	cd pro
	source setup.sh
	 \end{verbatim}
	 This setup.sh can be used from .bashrc.\\
	 The setup.sh should be effective also in running the e14 library.
   \item Make
	 \begin{verbatim}
	cd e14
	make install-headers
	 \end{verbatim}
	 Please ignore some messages at this stage.
	 \begin{verbatim}
	make
	 \end{verbatim}
   \item Terminate ssh tunnel.
	 \begin{verbatim}
	./scripts/e14connection stop
	 \end{verbatim}
	 You can terminate your ssh tunnel connection.
  \end{enumerate}
 \section{Package (Example package, novice01)}
   \begin{verbatim}
        novice01
        |-- Makefile        
        |-- REQUIREMENTS    Necessary for the PACKAGE.
        |-- novice01        For Header files.
        |   `-- func.h
        `-- src             For souce files.
            `-- fun.cc	
   \end{verbatim}
   \begin{enumerate}
    \item REQUIREMENTS is necessary for each package.\\
	  In other words, if a directory contains REQUIREMENTS,
	  the directory is a package.
	  Inside REQUIREMENTS
	  \begin{verbatim}
	#config      ---> If an entry begin with #, it is indeed
	                  a required library, but it is not a library.
	MidasManager ---> This is required libraray and will be linked.
	......
	  \end{verbatim}
    \item PACKAGES\\
          If PACKAGES is exist, make is done recursively for the packages
	  written in the PACKAGES.
    \item \$E14\_TOP\_DIR/e14/EXPORTED\\
	  If some packages in e14 is exported,
	  they are listed in \$E14\_TOP\_DIR/e14/EXPORTED.
          If e14 (top of all the packages) is exported,
	  \$E14\_TOP\_DIR/e14/SUBDIRS is made and
	  used instead.
   \end{enumerate}
 \section{Makefiles}
   \begin{enumerate}
    \item Target used in E14 libraray.
	  \begin{verbatim}
	install-headers:
	install-libs:
	install-bins:
	pre::
	distclean:
	clean::
	libs : 
	bins :
	  \end{verbatim}
	  You can add ``pre::'' and ``clean::'' at anywhere in order to
	  add some function. ``pre::'' is processed before ``install-headers''.
    \item \$E14\_CONFIG\_DIR/Makefile.e14\\
	  libs and bins are not defined, which should be written.
	  Others are applied recursively to PACKAGES.
	  \begin{verbatim}
	PACKAGE      =     # or package name, as PACKAGE = $(shell basename `pwd`)
	MAINS        =     # or exe name
	INSTALL_LIBS =     # or no
	INSTALL_BINS =     # or no
	include $(E14_CONFIG_DIR)/Makefile.e14
	libs:  ---> should define yourself
	bins:  ---> should define yourself	
	  \end{verbatim}
    \item \$E14\_CONFIG\_DIR/Makefile.bin\\
	  All targets are defined, which is standard
	  for libraries without geant4.
	  libs and bins are defined in addition to Makefile.e14.
	  \begin{verbatim}
	PACKAGE      =     # or package name, as PACKAGE = $(shell basename `pwd`)
	MAINS        =     # or exe names ---> multiple name can be written.
	                   # If MAINS=a b, then a.cc and b.cc with main()
	                   #      should be in package dir.
	INSTALL_LIBS =     # or no
	INSTALL_BINS =     # or no
	include $(E14_CONFIG_DIR)/Makefile.bin
	  \end{verbatim}
    \item \$E14\_CONFIG\_DIR/Makefile.g4\\
	  All targets are defined,
	  which support Geant4/CLHEP/ROOT.
	  libs and bins are defined in addition to Makefile.e14.
	  \begin{verbatim}
	PACKAGE      =     # or package name, as PACKAGE = $(shell basename `pwd`)
	MAINS        =     # or exe name ---> single name is allowed.
	                   # If MAINS=a, then a.cc with main()
	                   #     should be in package dir. 
	INSTALL_LIBS =     # or no
	INSTALL_BINS =     # or no
	include $(E14_CONFIG_DIR)/Makefile.g4		
	  \end{verbatim}
   \end{enumerate}
 \section{Web}
   You can look through e14 library from
   \begin{verbatim}
	http://chiaki.kek.jp/work/e14lib/svn/	
   \end{verbatim}
   You can see the repository from
   \begin{verbatim}
	http://chiaki.kek.jp/~e14rs/WebSVN/
   \end{verbatim}

 \section{Contact}
   Questions, comments, suggestions or requests are welcome.
   Please contact Hajime NANJO (nanjo@scphys.kyoto-u.ac.jp) .
   
  \newpage 
  \appendix
  \section{Setups for subversion}
  Many recent Linux distribution have subversion already.\\
  At RSC, for bash user
  \begin{verbatim}
	SVN=/e14/local/subversion/1.1.4
	export LD_LIBRARY_PATH=${SVN}/lib:${LD_LIBRARY_PATH}
	export PATH=${SVN}/bin:${PATH}
	export MANPATH=${SVN}/man:${MANPATH}
  \end{verbatim}
  At KEK CC, for bash user
  \begin{verbatim}
	SVN=/nfs/g/ps/klea/e14/local/subversion/1.1.4
	export LD_LIBRARY_PATH=${SVN}/lib:${LD_LIBRARY_PATH}
	export PATH=${SVN}/bin:${PATH}
	export MANPATH=${SVN}/man:${MANPATH}
  \end{verbatim}

\end{document}
