\documentclass[a4paper,12pt]{article}
\usepackage{cite}
\usepackage[dvips]{color}
\usepackage[dvips]{graphicx}
\usepackage{fancyhdr}
\usepackage{here}
\usepackage{threeparttable}
\usepackage{multirow}
\usepackage{amssymb}
\usepackage{amsmath}
\usepackage{mathrsfs}
\usepackage[FIGTOPCAP,nooneline]{subfigure}
\usepackage{boxedminipage}
\usepackage{ascmac}
\usepackage{multicol}


\setlength{\topmargin}{-5pt}
\setlength{\oddsidemargin}{0cm}
\setlength{\evensidemargin}{0cm}
\setlength{\textwidth}{16cm}
\setlength{\textheight}{24cm}
\setlength{\belowcaptionskip}{5pt}
\setcounter{tocdepth}{6}

\pagestyle{fancy}
\fancyhf{}
%\fancyhead[er]{\fancyplain{}{\sl\bf\leftmark}}
%\fancyhead[ol]{\fancyplain{}{\sl\bf\rightmark}}
\fancyhead[er]{{\sl\bf\leftmark}}
\fancyhead[ol]{{\sl\bf\rightmark}}
\fancyhead[el,or]{\rm\bf\thepage}
\fancypagestyle{plain} {
\fancyhead[er,ol]{}
\fancyhead[el,or]{\rm\bf\thepage}}
\renewcommand{\headrulewidth}{0.5pt}


\newenvironment{widelineskip}{\addtolength{\baselineskip}{24pt}}{\relax}
\begin{document}
\title{E14 Fast Simulator Manual}
\author{Hajime Nanjo}
\maketitle
 \section{Introduction}
  \subsection{What is the fast simulation}
  The detector responses to each particle are modeled with some 
  functions instead of simulating the energy loss, shower development and
  so on. The model functions are prepared using almost full
  simulations. Inefficiency weights for each particles and fusion
  weights for CsI clusters are calculated with such model functions. The
  energy and timing resolutions are also assinged to the CsI clusters.
  \subsection{Procedure}
  Three steps are exits for the simulation and analysis.
  \begin{itemize}
   \item Step1:gsim4test (E14 detector with fastSimulationLevel==6)\\
	 The charged veto systems and support strucutres are moved to far
	 upstream and all the materials are set to be sensitive
	 volumes. Then particles are stooped at the surfaces of the
	 detector volumes and such information are stored.
   \item Step2:e14fsim (\$E14\_TOP\_DIR/examples/e14fsim)\\
	 Inefficiency weights are assigned for all particles.
	 Fusion weights are assienged to gammas at CsI.
	 Clustering are performed for all the combinations of hits at
	 the CsI calorimeter. The energy and timing smearings are
	 also applied to the clusters.
   \item Step3:e14fsimg2ana (\$E14\_TOP\_DIR/examples/e14fsimg2ana)\\
	 Two gamma anaysis is performed for the two cluster events. They 
	 are treated as two gammas and the decay vertex is calculated
	 assuming the $\pi^0$ mass. Variouts cuts are applied and some
	 historgrams are filled.
  \end{itemize}

  \section{Step1 : gsim4test}
  \begin{verbatim}
	/GsimPhysicsListFactory/QGSP/buildAndRegister
/GsimPrimaryGeneratorActionFactory/GsimGeneralParticleSource/buildAndRegister gps
/gps/particle KLpi0nunu
/gps/energy 1 GeV
/gps/position 0 0 -1 m
/gps/direction 0 0 1

#/control/execute oglsini.mac
#/GsimEventAction/setVisualizationMode 1

/GsimDetector/world/setParameters 5*m 5*m 60*m
/GsimDetector/world/setOuterVisibility 0
/GsimDetectorFactory/GsimE14/buildAndRegister e14 /world

/GsimDetectorFactory/GsimTube/buildAndRegister os0 /world/e14 0 0  29.5*m+1./2.*cm  0 0 0
/GsimDetectorFactory/GsimTube/buildAndRegister os1 /world/e14 0 0 -29.5*m-1./2.*cm  0 0 0
/GsimDetectorFactory/GsimTube/buildAndRegister os2 /world/e14 0 0 0                 0 0 0
/GsimDetector/world/e14/os0/setParameters 0 2.41*m 1*cm 0 360*deg
/GsimDetector/world/e14/os1/setParameters 0 2.41*m 1*cm 0 360*deg
/GsimDetector/world/e14/os2/setParameters 2.4*m 2.41*m 59*m 0 360*deg
/GsimDetector/world/e14/os0/setOuterMaterial G4_Galactic
/GsimDetector/world/e14/os1/setOuterMaterial G4_Galactic
/GsimDetector/world/e14/os2/setOuterMaterial G4_Galactic
/GsimDetector/world/e14/os0/setSensitiveDetectorWithName os 0
/GsimDetector/world/e14/os1/setSensitiveDetectorWithName os 1
/GsimDetector/world/e14/os2/setSensitiveDetectorWithName os 2
/GsimDetector/world/e14/os0/setOuterVisibility 0
/GsimDetector/world/e14/os1/setOuterVisibility 0
/GsimDetector/world/e14/os2/setOuterVisibility 0

/GsimDetector/world/e14/setFastSimulationLevel 6

/GsimSpectrumFactory/GsimE14Spectrum/buildAndRegister
/GsimSpectrum/GsimE14Spectrum/addSpectrum Special

/GsimDetector/world/e14/setThisAndDaughterHitsStore true

/GsimDetectorManager/update

  \end{verbatim}
  The mbn, bcv, cv, and, bhcv are moved to far upstream when
  fastSimulationLevel==6. Support structures (CsI cylinder and endcap)
  are also moved to far upstream. All the other volumes are set
  sensitive. All particles are stopped at the surfaces of the volumes
  and such hit information (particle's PID and 4-vector) are stored.

  The gsim4test can be run as follows.
  \begin{verbatim}
	$E14_TOP_DIR//examples/gsim4test/bin/gsim4test e14toy.mac kpinn.root 100000 1 1
  \end{verbatim}
  
  \section{Step2 : e14fsim}
  Some branches belows in TTree made in gsim4test are used.
  \begin{itemize}
   \item GenParticle.\\
	 K, $\pi^0$ information.
   \item CC01.--CC06., CBAR.,FBAR.,CSI.,BHPV.   \\
	 Detector hit data. 
  \end{itemize}

  Hits in veto detectors are converted to E14FsimVeto. Inefficiency
  functions are used for the weight calculations.  
  Hits which are not $\gamma$ at CSI are also converted to E14FsimVeto.
  Gamma hits at CsI are convertred to E14FsimMCGamma.
  All clustering pattern including inefficiency and fusion effects are
  generated according as the number of E14FsimMCGamma. Data are stored
  cluster pattern by pattern. Inefficiency and fusion weights are
  assigned to each cluter pattern. Such products are ClusterWeight,
  where some of ClusterWeight through all the cluster pattern should be
  the number of events generated. All patterns with triple or more
  fusion are integrated as one pattern with one cluster and
  ClusterEffi==-1.  E14FsimClusters are created from such cluster
  pattern and energy and timing smearing are applied.
    

%   CC:cc01,cc02,cc03,cc04,cc05,cc06:DetectorEventData->Hit->pid,P,BARWeight
%   BAR:cbar,fbar:DetectorEventData->Hit->pid,P,BARWeight
%   CSI:csi
%   .   gamma
%   .     FsimMCGamma--->nGamma-->clusterPattern
%   .                                inefiGamma-->ineffi.weight-->veto
%   .                                cluster-->mcGamma
%   .     save clusterPattern by pattern:
%   .                nCluster==1 if tripple or more fusion
%   .   other
%   .     CSIWeight->veto
%   BHPV:bhpv
%   .     ineffi.
%   ..........................................
%   Discription for inefficiency functions
%   Discription for stored variables.
%   Energy and timing smearing
%.............................................
  \begin{itemize}
   \item KDecayMome[3]/D
   \item KDecayPos[3]/D
   \item KDecayTime/D
   \item Pi0number/I
   \item Pi0Mome[Pi0number][3]/D
   \item Pi0Pos[Pi0number][3]/D
   \item Pi0Time[Pi0number]/D
   \item eventID/I
   \item nHitCluster/I
   \item ClusterHitPos[nHitCluster][3]/D
   \item ClusterHitTotE[nHitCluster]/D
   \item ClusterHitAng[nHitCluster][2]/D
   \item ClusterHitTime[nHitCluster]/D
   \item ClusterHitParentID[nHitCluster]/I
   \item ClusterHitTrueP[nHitCluster][3]/D
   \item ClusterHitSigmaE[nHitCluster]/D
   \item ClusterHitSigmaXY[nHitCluster]/D
   \item ClusterEffi/D
   \item CalHitClusterWeight/D
   \item CalIneffiGammaWeight/D
   \item ClusterFusionProb/D
   \item ClusterWeight/D
   \item EventWeight/D
   \item mcEventID/I
   \item nHitVeto/I
   \item VetoPID[nHitVeto]/I
   \item VetoDetID[nHitVeto]/I
   \item VetoPos[nHitVeto][3]/D
   \item VetoMome[nHitVeto][3]/D
   \item VetoIneffi[nHitVeto]/D
   \item DetVetoWeight[10]/D
  \end{itemize}
  \section{Step3 : e14fsimg2ana}
  \begin{itemize}
   \item EventID/I
   \item OrigEventID/I
   \item mcEventID/I
   \item Pi0number/I
   \item Pi0Mome[Pi0number][3]/D
   \item Pi0Pos[Pi0number][3]/D
   \item nHitCluster/I
   \item ClusterHitTotE[nHitCluster]/D
   \item ClusterHitPos[nHitCluster][3]/D
   \item ClusterHitParentID[nHitCluster]/I
   \item ClusterEffi/D
   \item ClusterFusionProb/D
   \item ClusterWeight/D
   \item EventWeight/D
   \item nHitVeto/I
   \item VetoPID[nHitVeto]/I
   \item VetoDetID[nHitVeto]/I
   \item VetoPos[nHitVeto][3]/D
   \item VetoMome[nHitVeto][3]/D
   \item VetoIneffi[nHitVeto]/D
   \item VetoWeight/D
   \item CutCondition/I
   \item RecEvenOdd/I
   \item RecVertZ/D
   \item RecPi0Pt/D
   \item RecVertZsig2/D
   \item RecPi0Mass/D
   \item RecPi0E/D
   \item RecPi0Px/D
   \item RecPi0Py/D
   \item RecPi0Pz/D
   \item RecPi0Vx/D
   \item RecPi0Vy/D
   \item RecPi0Vz/D
  \end{itemize}

  The cut lists.
  \begin{enumerate}
   \item Energy Cut : iCut += 0x0001\\
	 Egamma\_Min=0.1 GeV\\
	 Egamma\_Max=2.0 GeV
   \item Fiducial in CsI : iCut += 0x0002\\
	 R\_Min=175 mm\\
	 R\_Max=850 mm
   \item Vertex Cut : iCut += 0x0004\\
	 Zvert\_Min=3000 mm\\
	 Zvert\_Max=5000 mm
   \item PT Cut : iCut += 0x0008\\
	 Pt\_Min=0.12 GeV\\
	 Pt\_Max=0.25 GeV
   \item Acp\_angle  : iCut += 0x0010\\
	 Acop\_Min=30 deg
   \item Two gamma distance : iCut += 0x0020\\
	 gDist\_Min=300 mm
   \item Total energy of two gamma : iCut += 0x0040\\
	 TotE\_Min=0.5 GeV
   \item In\_ang cut : Odd-pair cut : iCut +=0x0080\\
	 Etheta\_Min=2.5 GeV
   \item E\_ratio cut  : iCut +=0x0100\\
	 Eratio\_Min=0.2
   \item Pi0 Kinematic cut : iCut +=0x0200\\
  \end{enumerate}

  The detector ID lists.
  \begin{table}[h]
   \begin{tabular}{c|c}
    Detector&VetoDetID \\
    CC01&1 \\
    CC02&2 \\
    CC03&3 \\
    CC04&4 \\
    CC05&5 \\
    CC06&6 \\
    CBAR&7 \\
    FBAR&8 \\
    CSI&9 \\
    BHPV&10 
   \end{tabular}
  \end{table}

\end{document}
